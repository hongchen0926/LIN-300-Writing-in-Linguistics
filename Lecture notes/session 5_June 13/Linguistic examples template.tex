\documentclass{article}
\usepackage{textglos} % good semantic markup for inline examples, like \xv{...}, \xm{...}, etc.
%\usepackage{fixltx2e} % only needed if you have TeX Live < 2015
\newcommand*{\IND}[1]{\textsubscript{#1}}
\usepackage{gb4e}
\noautomath % you should always declare this after loading gb4e

\usepackage{forest}%% for linguistic trees

\usepackage{tikz} % for linguistic trees
\usepackage{tikz-qtree, ,tikz-qtree-compat} % for linguistic trees
\tikzset{every tree node/.style={align=center, anchor=north}}% set up the arrows 

\usepackage{tipa} % for IPA
\usepackage{phonrule} % for ponology rule


\usepackage{amsmath}	% for semantic representation
\usepackage{stmaryrd}	% for semantic representation [[ ]]
\newcommand{\evaluation}[2][]{\ensuremath{\llbracket #2\rrbracket^{#1}}}	% semantic representation [[ ]]

\begin{document}


\section{Basic Linguistic examples}


\begin{exe}
\ex[?]{His mother loves every boy no matter what.}
\label{ex:questionable-English} %give the sentence a name, later you can cite it. 
\end{exe}

\begin{exe}
\ex[]{Strong crossover
\begin{xlist}
\ex[*]{He loves everyone.} %%% to make the X_i index, you can use X\IND{i} or X_{i}, but it doesn't work for the Overleaf complier. 
\ex[*]{She thinks everyone is smart}
\label{ex:everyone-is-smart}
\end{xlist}
}
\label{ex:strong-crossover}
\end{exe}

According to native speakers, (\ref{ex:questionable-English}) is marginally acceptable. %%\ref{} is the command for cross-reference

The examples in (\ref{ex:strong-crossover}) exemplify the phenomenon of strong crossover.
For example, in (\ref{ex:everyone-is-smart}), \textit{she} c-commands \textit{everyone}.
However, pronouns cannot c-command their binders.


\section{Glossing examples}
\begin{exe}
\ex\label{korean example} This is a Korean example. 
\gll Nwukwu-na ku mwuncey-lul phwu-ess-ta.\\
		who-\textsc{na} that problem-\textsc{acc} solve-\textsc{past}-\textsc{decl}\\
\glt `Everyone solved that problem.'
\end{exe}


\section{Typsetting trees with forest package}
For drawing a tree, line breaks are not necessary. You could have produced the same output by writing [VP [V] [DP] ].\\
\begin{exe}
\ex{
\begin{forest}
[\textcolor{red}{VP}
[\textcolor{blue}{V}]
[\textcolor{blue}{DP}]
]
\end{forest}
}
\end{exe}


\section{Typsetting trees with tikz packages}
\begin{exe}
\ex{
\Tree	[.S
							[.NP 
								[.N John ] 
							]
							[.VP 
								[.V loves ]
								[.NP 
									[.N Mary ] 
								]
							]
						]\\
}
\end{exe}

You center a tree by using ``center'' command.
\begin{exe}
\ex{
                            \begin{center}
                                  \Tree [.S [.NP LaTeX ] [.VP [.V is ] [.NP fun ] ] ]
                            \end{center}
}
\end{exe}

\section{Typsetting trees with movements}
 You can draw tree with movement.\\
  \begin{tikzpicture} % this means that you can call command for drawing curves
      \Tree	[.S
                 [.\node(M){Mary_1}; ]
                 [.S
							[.NP 
								[.N John ] 
							]
							[.VP 
								[.V loves ]
								[.NP 
									[.N \node(Mtrace){$<$t_1$>$}; ] 
								]
							]
						]
                                        ]\\
 % \draw[->] (Mtrace) [in=-90,out=-90,looseness=1.5]  to (M); %you also can assgin looseness=1.0 or 0.5. It depends on you.
 %\draw[->] (Mtrace) [in=-120,out=-100,looseness=1, red]  to (M); % you also can change the color of the row
\draw[->] (Mtrace) [in=-120,out=-100,looseness=1, dashed,red]  to (M); %you also can change the shape of the row
 \end{tikzpicture}



\section{Typsetting IPA}
You can make cool IPA fonts in \LaTeX{} with the \textit{tipa} package.\\
 % add packages for ipa
                             \begin{center}
                                  \textipa{abcdefghijklmnopqrstuvwxyz}          \\
                                       \textipa{ABCDEFGHIJKLMNOPQRSTUVWXYZ}\\    

                                      \textipa{! * + = ? . , / [ ] ( ) ` ' | ||}    \\   %% symbols
                                      \textipa{1234567890 @}                       \\  %% vowels
                                    \textipa{\;A \;B \;E \;G \;H \;I \;L \;R \;Y} \\  
                                    \textipa{\:d \:l \:n \:r \:s \:t \:z}         \\
                                     \textipa{\!b \!d \!g \!j \!G \!o} \\
                             \end{center} 

                          
 You can make really pretty phonological rules too!\\
\begin{center}
%add package for phonology rule
\phonb{\phonfeat{+stop \\ +consonant \\ +alveolar} }{\textipa{R}}{\phonfeat{+vowel \\ +stressed} }{\phonfeat{+vowel \\ +stressed} }\\
 \end{center}


\section{Typsetting Semantics}
You can write semantic equations more easily.\\
\begin{exe}
\ex \evaluation{X\mbox{-}na \ Q} = $ \forall x_i [(x_i \in X) \supset Q(x_i)] $  where X = $ \{x_1, x_2, \cdots x_n\} $\\

\ex $\exists x [white(x) \& dog(x)]$ \\
\ex $\forall x [linguist(x) \rightarrow know(x, $\LaTeX$)]$
\end{exe}




\end{document}